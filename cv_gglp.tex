%!TEX TS-program = xelatex
\documentclass[a3paper]{adcv_color}
\usepackage[english]{babel}
\usepackage{multicol}
\usepackage{scrextend}
\setlength{\columnsep}{1cm}
\newcommand*{\tuv}{\href{https://www2.acom.ucar.edu/modeling/tropospheric-ultraviolet-and-visible-tuv-radiation-model}{\textbf{TUV }}}
\newcommand*{\aeronet}{\href{https://aeronet.gsfc.nasa.gov/}{\textbf{AERONET}}}
\newcommand*{\sedema}{\href{https://www.sedema.cdmx.gob.mx/}{\textbf{SEDEMA}}}
\newcommand*{\omi}{\href{https://aura.gsfc.nasa.gov/omi.html}{\textbf{OMI}}}

\title{Gamaliel-Lopez CV}
\pagestyle{empty}
\adcvname{López Padilla}{Giovanni Gamaliel}{}
\adcvemail{giovannilopez9808}{gmail}{com}
\adcvphone{(+52) 871-278-37-70}
\adcvwebsite{http://resumegglp.herokuapp.com/}{resumegglp.herokuapp.com}
% \adcvdate{}

\begin{document}
\changefontsizes{14.5pt}
I like science and I like to approach it from an approach in which problems can be solved with the optimal implementation of IT to obtain a better understanding of the phenomena. In the same way I feel a big curiosity for the things that surround me so sometimes I get to disarm 
to see how it works inside, and to see what procedures or steps can be improved.
\begin{multicols}{2}
\section{Projects}

\begin{minipage}{0.8\linewidth}
  \textbf{Estimate of solar exposure time for obtaining Pre Vitamin D in Rosario, Argentina}\\
  Pending publication
\end{minipage}
  \begin{minipage}{0.2\linewidth}
    \begin{flushright}
    \vspace{-0.8cm}
    december 2020
    \end{flushright}
  \end{minipage}\\

  With the \tuv model and ozone column measurements provided by the instrument \omi-NASA, python code was created, which
  I consulted the HDF files of \omi~and created a resulting file which read \tuv model for obtain the erythemal irradiance data
   on Rosario, Argentina, with another code that read these results and calculated the solar exposure time for 
   the production of pre-vitamin D in the human body, taking into account the exposure limits depending on the phototype.
    These results were stored within a csv file
 \\

  \begin{minipage}{0.8\linewidth}
    \textbf{Analysis in the trend of the UV index in Mexico City in the period 2000-2019}\\ 
    Pending publication
  \end{minipage}
    \begin{minipage}{0.2\linewidth}
      \begin{flushright}
        \vspace{-0.7cm}
      october\\ 2020
      \end{flushright}
    \end{minipage}\\
    
    With the measurement history of the meteorological stations of the Secretary of the Environment (\sedema)~of Mexico City,
    the data obtained from the robotic network of aerosols (\aeronet) and the NASA \omi~instrument, i do a study on the trend of atmospheric components
     in the period 2000-2019, finding results about the measures that have been taken in the reduction of the atmospheric pollution in the region.
     All this was calculated with python and fortran codes for statistical analysis and graphs.\\

  \begin{minipage}{0.8\linewidth}
    \textbf{Location of fires in the Parana River Delta}\\
    \href{https://github.com/giovannilopez9808/Posters_templates/blob/master/Informe\%20IFIR/Focos\%20de\%20incendio\%20en\%20las\%20islas\%20del\%20Paran\%C3\%A1\%20\%20frente\%20a\%20Rosario\%20\%20(Piacentini\%2C\%20Ipi\%C3\%B1a\%2C\%20Lopez-Padilla\%2C\%20Bolmaro)\%20(Piacentini).pdf}{\textbf{Report delivered to the government of Rosario, Argentina}}
  \end{minipage}
    \begin{minipage}{0.2\linewidth}
      \vspace{-0.7cm}
      \begin{flushright}
        august
        \\2020
      \end{flushright}
    \end{minipage}\\
  
    Through satellite's data provided by NASA and NOA's plataform \href{https://firms.modaps.eosdis.nasa.gov/}{\textbf{FIRMS}}, i obtain information about the
    number of fires that was measurement by three satellite's instruments: Terra, Aqua and VIIRS. Based on these satellite's data a python code was created
    to process and count the number of fires that affected the region of Rosario and surrounding areas, in the period from June 3 to August 4, 2020.\vspace{1cm}\\
  \begin{minipage}{0.8\linewidth}
    \textbf{Calculation of solar exposure time for Psoriasis treatment in Mexico City}\\
    \href{https://github.com/giovannilopez9808/Posters_templates/blob/master/2020/CNF/TES/main.pdf}{\textbf{Poster}}, \href{http://tes-v1.herokuapp.com/}{\textbf{Plataforma}}
  \end{minipage}
    \begin{minipage}{0.2\linewidth}
      \vspace{-0.9cm}
      \begin{flushright}
      august \\2020
      \end{flushright}
    \end{minipage}\\

    Based on the measurements of erythemic solar radiation provided by the Secretary of the Environment of the 
    Mexico City (\sedema). the solar exposure time (TES) was calculated taking into account the characteristics for different phototypes, different dates and sky conditions.
     A code was made to read the measurements and calculations of the TES, in conjunction with the Easter Dermatological Center 
    a platform called \textit{TES para tu salud} was created where patients can consult the time they need by entering the parameters specified above. This platform was developed as a
    web page making one of the python Flask library, HTML and CSS code.\\

  \begin{minipage}{0.8\linewidth}
    \textbf{Analysis of UVA and erythemic irradiances measured by the Mexico City Atmospheric Monitoring System}\\
    \href{https://github.com/giovannilopez9808/Posters_templates/blob/master/2019/AFA/Analisis\%20indice\%20UV/Analisis\%20de\%20irradiancia.pdf}{\textbf{Poster}}
  \end{minipage}
    \begin{minipage}{0.2\linewidth}
      \vspace{-0.9cm}
      \begin{flushright}
      october \\2019
      \end{flushright}
    \end{minipage}\\

    With the satellite's measurements made by the NASA \omi~instrument, the Aerosol Robotic Network (\aeronet). i It was verified how close the results of the model \tuv are to reality, this model was modified in its source code so that it could be automated, because originally you can only run for one day with only one data entry.\\

  \begin{minipage}{0.8\linewidth}
    \textbf{Study of solar irradiance with two radiative transfer models}\\
    \href{https://github.com/giovannilopez9808/Posters_templates/tree/master/2019/CNF/Transferencia\%20radiativa}{\textbf{Poster}}
  \end{minipage}
    \begin{minipage}{0.2\linewidth}
      \vspace{-0.9cm}
      \begin{flushright}
      august \\2019
      \end{flushright}
    \end{minipage}\\

    Using the \tuv model and the
    \href{https://www.nrel.gov/grid/solar-resource/smarts.html}{\textbf{SMARTS}} model, both written in the Fortran language, we analyzed the relativistic differences in
      results for the Monterrey metropolitan area with the same parameters. The source codes of the two models were modified to read the values of the parameters 
      from a database and this automate the calculation of solar irradiances which were analyzed with a code written in python.
  \end{multicols}
  \begin{multicols}{2}

\section{Hability}

\begin{minipage}{0.25\linewidth}
  \begin{flushright}
    \textbf{Programaming}\\
    \textbf{Libraries}
    \vspace{0.4cm}\\
    \textbf{Hardware}\\
    \textbf{Web}\\
    \textbf{Languages}
  \end{flushright}
\end{minipage}
\hspace{0.3cm}
\begin{minipage}{0.75\linewidth}
  \vspace{0.25cm}
  Python, Fortran, LaTeX, Java, Julia\\
  Numpy, Matplotlib, Pandas, Scipy, Flask, sqlite\\
  Arduino\\
  CSS, HTML5\\
  Spanish, english
\end{minipage}
\vspace{2cm}\\
\section{Education}

\textbf{Bachelor of Science in Physics}, Universidad Autónoma de Nuevo León\\

\section{Courses}

\begin{minipage}{0.7\linewidth}
  \textbf{Qubit x Qubit semestre 1}\\
  \textbf{Partial differential equations from theory to numerical approximations}\\
  \textbf{Qiskit Globar Summer School, IBM}
\end{minipage}
\begin{minipage}{0.3\linewidth}
  %\vspace{-0.cm}
  \begin{flushright}
  august 2020\\
  august 2020\vspace{1cm}\\

  july 2020
\end{flushright}
\end{minipage}
  % \textbf{Qubit x Qubit semestre 2}diciembre 2020
  
  % \textbf{Qubit x Qubit semestre 1} agosto 2020
  
  %  agosto 2020
  
  % \textbf{Qiskit Globar Summer School, IBM} julio 2020
  
  % \textbf{Escuela de Verano, CIMAT Merida} junio 2020
  
  % \textbf{Escuela Avanzada en Física, Cinvestav} julio 2019
  
  % \textbf{XXVII Escuela de Verano en Física, UNAM} junio 2019


\end{multicols}
\end{document}
