\documentclass[a3paper]{adcv_color}
\usepackage[english]{babel}
\usepackage{multicol}
\usepackage{scrextend}
\setlength\itemsep{0em}
\title{Gamaliel-Lopez-Padilla}
\pagestyle{empty}
\adcvname{Giovanni Gamaliel}{López Padilla}{Machine Learning Engineer | Data Scientist | Statistican}
\adcvemail{giovannilopez9808}{gmail}{com}
\adcvphone{8712783770}
\adcvwebsite{https://giovannilopez9808.github.io/}{giovannilopez9808.github.io}
\adcvgithub{https://github.com/giovannilopez9808/}{giovannilopez9808}
\adcvlinkedin{https://github.com/giovannilopez9808/}{giovannilopez9808}
\newcommand{\work}[3]{
	\begin{minipage}{0.75\linewidth}
		\textbf{#1} | #2
	\end{minipage}
	\begin{minipage}{0.25\linewidth}
		\begin{flushright}
			\textit{#3}
		\end{flushright}
	\end{minipage}
	\vspace{-1cm}
}
\begin{document}
\changefontsizes{14pt}
\section{Perfil}\\
\begin{addmargin}[0.5em]{0.2em}
	\begin{flushleft}
		Me gusta la ciencia y como la comunidad realiza  aportaciones con sus investigaciones para resolver algún problema usando resultados expuestos de otras personas para lograr sus objetivos. De la misma forma me gusta la filosofía del open source, donde las personas publican sus proyectos y cualquier persona puede aportar y usar sus avances.\\
	\end{flushleft}
\end{addmargin}

\section{Experiencia}\\
\begin{addmargin}[0.5em]{0.5em}
	\work{Consultor senior de inteligencia artificial}{Algorithia}{Junio 2024 - Actualidad}
	\begin{itemize}
		\item Implemente modelos basados en deep learning utilizando datos historicos de los clientes con el fin de prevenir fraudes y aumentar la eficiencia de cobranza realizada por el Banco Azteca. El resultado de estos mdoelos se reflejo en un aumento de 2.5 millones de pesos.
		\item Se tuvieron reuniones con departamentos internos y empresas externas con el fin de aumentar la comunicación entre equipos y recolectar ideas para generar nuevos proyectos. Esto ayudo a tener un ambiente de confianza entre los miembros de todos los equipos.
	\end{itemize}
	\work{Ingeniero de aprendizaje automático}{CIMAT}{Agosto 2021 - Junio 2023}
	\begin{itemize}
		\item Desarrollé diversos modelos basados en machine learning y deep learning implementados usando librerías de Python como scikit-learn, TensorFlow, PyTorch y NumPy.
		\item Diseñe flujos de información para administrar el funcionamiento, entrenamiento y validación de los modelos de redes neuronales en una red local y en servicios en la nube como AWS.
	\end{itemize}
	\work{Análista de datos}{UANL}{Agoso 2018 - Julio 2021}
	\begin{itemize}
		\item Realice investigaciones y análisis a datos relacionados con radiación solar, componentes atmosféricos y valores geográficos recabados por estaciones meteorológicas y estimaciones satelitales.
		\item Los proyectos han sido publicados en artículos internacionales, congresos y notas para el público general. Algunos resultados obtenidos fueron expuestos a otros expertos para la toma de decisiones en la regularización de emisión de contaminantes a la atmósfera.
	\end{itemize}
	%\work{Profesor de matemáticas, análisis de datos y aprendizaje automático}{UANL-CIMAT}{Agosto 2017 - Enero 2024}
	%\begin{itemize}
	%\item Impartí clases como profesor y profesor auxiliar en clases de matemáticas y computación a estudiantes de nivel licenciatura, maestría y doctorado.
	%\item Prepare notas de apoyo, ejercicios de práctica y exposiciones semanales para explicar de forma didáctica. Además de dar retroalimentación para mejorar su comprensión del tema.
	%\end{itemize}
\end{addmargin}

\section{Proyectos}\\
\begin{addmargin}[0.5em]{0.5em}
	\subsection{Pronóstico diario de índice UV basado en mediciones in situ en el Área Metropolitana de Monterrey}
	\begin{itemize}
		\item Cree un modelo basado en redes neuronales para obtener el pronóstico del índice UV para las siguientes 24 horas basado en mediciones in situ y datos satelitales.
		\item Debido a los resultados obtenidos en este estudio, se llevo acabo una reunión con el Sistema Integral del Monitoreo Ambiental para acordar la implementación del modelo en su sistema y comunicar sus estimaciones al público general.
	\end{itemize}
	\subsection{Clasificación de las condiciones del cielo por medio de mediciones de radiación solar global}
	\begin{itemize}
		\item Implemente un modelos basado en redes neuronales multicapa, convolucionales y recurrentes para estimar la condición del cielo dado un conjunto de mediciones in situ.
	\end{itemize}
	%\subsection{Ultraviolet Radiation Environment of a Tropical Megacity in Transition: Mexico City 2000–2019}
	%\begin{itemize}
		%\item Realice un análisis en las tendencias de los contaminantes atmosféricos por medio de modelos de regresión lineal y medias móviles.
		%\item Debido a las conclusiones expuestas en el articulo, se realizó una reunión con la secretaria de medio ambiente de la CDMX y académicos expertos en el área.
	%\end{itemize}
	\subsection{Detección autommática de tweets agresivos}
	\begin{itemize}
		\item Con base en el conjunto de datos de la competencia, MEX-A3T implementé un modelo de clasificación binaria basado en la arquitectura Transformer. Los resultados obtenidos superan en un 20\% el rendimiento expuesto por los demás participantes del concurso.
	\end{itemize}
	%\subsection{Segmentación de las vasos capilares de los ojos por medio del modelo pix2pix}
	%\begin{itemize}
		%\item Se desarrolló un modelo de redes neuronales basado en la arquitectura pix2pix con el fin de segmentar los vasos capilares de los ojos con base en imágenes médicas.
		%\item Se logró replicar los resultados expuestos en el artículo original del modelo y además se mejoró su rendimiento añadiendo un preprocesamiento a las imágenes dadas.
	%\end{itemize}
	%\subsection{Análisis de los patrones de uso de la aplicación MiBici en el periodo 2015-2018}{Noviembre 2021}{-0.2}{https://raw.githubusercontent.com/giovannilopez9808/MiBici/main/Document/Main.pdf}{Reporte}}
	%\subsection{

	%\subsection{Índice de marginacion en México dentro del periodo 2015-2020}{Octubre 2021}{-0.2}{https://raw.githubusercontent.com/giovannilopez9808/IndiceDeMarginacion2015_2020/main/Document/Main.pdf}{Reporte}

	%\subsection{Análisis de la radiación solar UV para la síntesis de Pre-Vitamina D en la piel en Rosario, Argentina}{Enero 2022}{-0.8}{https://anales.fisica.org.ar/journal/index.php/analesafa/article/view/2318}{Artículo}

	%\subsection{Data Science, Data Analyst}{Impacto en la calidad del aire en Rosario durante la quema de pastizales en el Delta del Rio de Paraná}
	%
	%\item Se realizó un estudio acerca del impacto de la calidad del aire en la ciudad de Rosario, Argentina y las consecuencias que tuvo con la población del área. El resultado de este estudio fue comunicado al gobierno del estado derivando en una investigación más regirusa desde diferntes puntos de vista a partir de datos in situ y satelitales.
	%}
	%\subsection{Estimación de los tiempos de exposición solar para obtener el tratamiento para la Psoriasis en CDMX}{Agosto 2020}{-2}{https://github.com/giovannilopez9808/Documents/raw/master/Posters/2020/CNF/TES/main.pdf}{Poster}

	%\subsection{Análisis de la irradiancia UVA y eritemica en la Ciudad de México}{Octubre 2019}{-0.2}{https://github.com/giovannilopez9808/Documents/raw/master/Posters/2019/AFA/Analisis\%20indice\%20UV/Analisis\%20de\%20irradiancia.pdf}{Poster}

	%\subsection{Análisis de la irradiancia solar con dos modelos de transferencia radiativa}{Agosto 2019}{0}{https://github.com/giovannilopez9808/Documents/raw/master/Posters/2019/CNF/Transferencia\%20radiativa/main.pdf}{Poster}
	%\end{multicols}
\end{addmargin}
\begin{minipage}{0.35\linewidth}
	\section{Educación}\\
	\begin{addmargin}[0.5em]{0.5em}
		\textbf{Licienciatura en Física} \\
		Universidad Autónoma de Nuevo León\\
		Nuevo León, Mexico.\\

		\textbf{Maestria en Ciencias con Especialidad en Computación y Matemáticas Industriales}\\
		Centro de Investigación en Matemáticas\\
		Guanajuato, México
	\end{addmargin}
\end{minipage}
\begin{minipage}{0.65\linewidth}
	\section{Habilidades y conocimientos}\\
	\begin{addmargin}[0.5em]{0.5em}
		\begin{itemize}[label={}]
			\setlength\itemsep{0em}
			\item \textbf{Idiomas}: Inglés (B2), Español (Nativo).
			\item \textbf{Blandas}:
				Comunicación, creatividad, pensamiento crítico.
			\item \textbf{Lenguages de programación}:
				R, C, C++, Python, Fortran, Java.
			\item \textbf{Librerias}:
				Tensorflow, Pytorch, Scikit-Learn, OpenCV, NTLK, Transformers, Numpy, Pandas.
			\item \textbf{Servicios en la nube}:
				AWS, Google Cloud, GitHub. Cluster computing.
			\item \textbf{Herramientas}:
				SQL, Git, GitHub, Linux, Power Bi, ETL.
		\end{itemize}
	\end{addmargin}
\end{minipage}
\end{document}
