%TEX TS-program = xelatex
\documentclass[a3paper]{adcv_color}
\usepackage[english]{babel}
\usepackage{multicol}
\usepackage{scrextend}
\setlength{\columnsep}{1cm}

\title{Gamaliel-Lopez-Padilla}
\pagestyle{empty}
\adcvname{López Padilla}{Giovanni Gamaliel}{}
\adcvemail{giovannilopez9808}{gmail}{com}
%\adcvphone{}
%\adcvwebsite{}
%\adcvphone{(+52) 871-278-37-70}
%\adcvwebsite{http://resumegglp.herokuapp.com/}{resumegglp.herokuapp.com}
% \adcvdate{}
\newcommand{\proyect}[3]{\begin{minipage}{1\linewidth}
    \begin{minipage}{0.8\linewidth}
      \textbf{#1}
    \end{minipage}
    \begin{minipage}{0.19\linewidth}
      \begin{flushright}
        \vspace{#3cm}
        \textit{#2}
      \end{flushright}
    \end{minipage}\\
  \end{minipage}
  \vspace{0.1cm}\\
}
\begin{document}
\changefontsizes{15.5pt}
I like science and the approaches that i can do with algorithms and code implementations to solve problems or analyze data. In the same way i have a big curiosity in the things that surround me. For these reason sometimes i disassemble or see how a framework or algorithm works inside.

\section{Projects}
\begin{multicols}{2}

  \proyect{\textbf{Ultraviolet Radiation Environment of a Tropical Megacity in Transition: Mexico City 2000–2019}}{August 2021}{-0.6}
  \href{https://pubs.acs.org/doi/10.1021/acs.est.0c08515}{\textbf{Article}}

  The Environment Secretary have been measuring in 13 meteorological stations since 1997. We used all the data in the period 2000-2022 to do a study about the trend of atmospheric components as suspended particles, ozone, CO, NO$_2$ and SO$_2$.\\

  \proyect{\textbf{Analysis of the UV solar irradiance for the synthesis of Pre-Vitamin D$_3$ on the skin in Rosario, Argentina}}{January 2022}{-0.8}
  \href{https://anales.fisica.org.ar/journal/index.php/analesafa/article/view/2318}{\textbf{Article}}

  Few  foods  contain  vitamina  D$_3$ naturally,  the  main  source  of  obtaining  the  ultraviolet (UV) solar radiation, which triggers the synthesis on the skin’s surface. In this study, the effective UV solar irradiance for the synthesis of pre-vitamin D$_3$ was determined in Rosario city, Argentina, using three methods: Proportionality Coefficient, Herman’s equation and TUV model.\\

  \proyect{\textbf{Impact on air quality in the Rosario city due to grassland burning in the Delta of Paraná River}}{August 2022}{-0.6}
  \href{https://rephip.unr.edu.ar/handle/2133/24201}{\textbf{Report}}

  Through satellite's data provided by NASA and NOA's plataform. The fire alarms were measurement by the VIIRS satellite instrument. We process the number of fires that affected the region of Rosario and surrounding areas. We correlate this information with the suspended particles and the number of deaths in the same period of time.\\

  \proyect{Cloud classification with radiacion solar measurements.}{June 2022}{0}
  \href{https://github.com/giovannilopez9808/Cloud_classification/raw/main/Document/Main.pdf}{\textbf{Report}}

  The cloud cover is an important characteristic to any type of weather forecasting. The solar radiation is a natural detector of clouds. In recent decades, a several types of models have been developed to classify different sky types according to cloud conditions or percentage cloud cover.

  \columnbreak

  \proyect{Estimation of solar exposure for Psoriasis treatment in Mexico city}{August 2020}{-0.2}
  \href{https://github.com/giovannilopez9808/Documents/raw/master/Posters/2020/CNF/TES/main.pdf}{\textbf{Poster}}, \href{http://tes-v1.herokuapp.com/}{\textbf{Website}}

  Based on the measurements of erythemal radiation provided by the Environment Secretary of Mexico City, the solar exposure time was estimated for all the different phototypes and cloud conditions. The results were communicated to the Easter Dermatological Center. The patients can consult their treatment in a website.\\

  \proyect{\textbf{Analysis of UVA and erythemic irradiances in Mexico City}}{October 2019}{-0.2}  \href{https://github.com/giovannilopez9808/Documents/raw/master/Posters/2019/AFA/Analisis%20indice%20UV/Analisis%20de%20irradiancia.pdf}{\textbf{Poster}}

    With the ozone measurements from OMI-NASA instrument and the AOD from AERONET, we verified the ratios from TUV model results respect the surface measurements. This results was submitted in 104 Reunión de la Asociación Física Argentina.\\

    \proyect{Analysis of solar irradiance with two transfer radiative models}{August 2019}{0}
    \href{https://github.com/giovannilopez9808/Documents/raw/master/Posters/2019/CNF/Transferencia%20radiativa/main.pdf}{\textbf{Poster}}

      The TUV model and the SMARTS model estimate the irradiance solar and solar spectre respectively. We analyzed the relativistic differences of the estimations for the Monterrey metropolitan area. The source codes of the two models were modified to read the values of the parameters from a database.\\


      \section{Education}\\

      \textbf{Bachelor of Science in Physics} \\
      Universidad Autónoma de Nuevo León\\
      Nuevo León, Mexico.\\

      \textbf{Master's degree in computer science}\\
      Centro de Investigación en Matemáticas\\
      Guanajuato, México
    \end{multicols}
    \begin{flushleft}
      \changefontsizes{10pt}
      \textbf{Note}:

      The words in blue color (article, report, poster and website) are hyperlinks.
    \end{flushleft}
    \end{document}
