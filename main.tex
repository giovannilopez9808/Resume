\documentclass[12pt,letterpaper]{article}
\input{Files/package.tex}
\input{Files/highlight.tex}
\input{Files/header.tex}
\setlength\itemsep{0em}
\begin{document}
\setmargins{2cm}
{1cm}
{1cm}
{25cm}
{10pt}
{1cm}
{0pt}
{2cm}
\subfile{Files/titulo.tex}
\textbf{Publicaciones (2)}
\begin{itemize}[label={}]
    \item Ipiña A., López-Padilla G., Fisanotti A. L., Dávalos M. y Piacentini R.D., 2022. \textit{Análisis de la irradiancia solar UV para la síntesis de pre-vitamina D3 en la piel}, en Rosario, Argentina. AAFA 32, 88-92.
    \item Ipiña A., López-Padilla G., Retama A., Piacentini R.D. and Madronich S., 2021. \textit{Ultraviolet Radiation Environment of a Tropical Megacity in Transition: Mexico City 2000−2019}. Environmental Science and Technology 55, 10946−10956
\end{itemize}
\textbf{Proceddings (2)}
\begin{itemize}[label={}]
    \item Ana Laura Fisanotti, Gamaliel López-Padilla y Adriana Ipiña Hernández. \textit{Influencia de la contaminación producida por los incendios del delta del río Paraná sobre los tiempos de exposición solar para la síntesis de vitamina D.} Artículo extenso \url{https://www.calameo.com/read/007442498347f083345e2} Ciencia in situ, pág. 29-35, Vol. 7 No. 1 ISSN 2469-2441, Dic 2022.
    \item Ana Laura Fisanotti, Gamaliel López-Padilla, Silvana Spinelli y Adriana Ipiña Hernández. Top-Ten CoCAEM XXXII Intercambiando Ciencia - Décimo puesto. Mención de resumen por la obtención del 10° lugar en el top-ten como mejor trabajo del XXXII Congreso Científico Argentino de Estudiantes de Medicina (CoCAEM). Mater Ciencia, pág. 24 Vol. 6 No. 2 ISSN 2469-2166, nov 2021.
\end{itemize}
\textbf{Desarrollos (2)}
\begin{itemize}[label={}]
    \item Desarrollo de herramienta computacional para el conteo de focos de incendios en dos regiones: la Sierra  Madre Oriental (México) y en el Delta del río Paraná (Argentina), 2021.
    \item Desarrollo de la plataforma \textit{TES para tu salud}  que determina los Tiempos de Exposición Solar adecuados para el tratamiento de Psoriasis en la Ciudad de México en Colaboración con el Centro Dermatológico Pascua, Ciudad de México, 15 may 2020. \url{https://tes-jve0.onrender.com}
\end{itemize}
\textbf{Premios y distinciones (2)}
\begin{itemize}[label={}]
    \item  Ganador del 4° lugar en el  entre las diez Mejores investigaciones 2022 presentadas en el XXXII Jornadas Científicas Anuales de la Asociación Civil Científica Rosarina de Estudiantes de la Salud (ACRES).
    \item Ganador del 10° lugar en el Top Ten 2021 de las mejores investigaciones presentadas en el Congreso Científico Argentino de Estudiantes de Medicina (CoCAEM XXXII) Facultad de Ciencias Médicas de la Universidad de Buenos Aires.
\end{itemize}
\textbf{Presentaciones de trabajos en Congresos/Reuniones/Jornadas/Encuentros (25)}

XIII Congreso Argentino de Toxicología

\begin{itemize}[label={}]
    \item \textit{Incendios forestales en el Delta del Paraná: impactos en la calidad del aire y la salud.} Adriana Ipiña, Gamaliel López-Padilla y Bettina Bongiovanni. Buenos Aires, 20-22 Sep. 2023
\end{itemize}

XXXII Jornadas Científicas Anuales de la Asociación Científica Rosarina de Estudiantes de la Salud (ACRES)
\begin{itemize}[label={}]
    \item \textit{Impacto en la mortalidad de los habitantes de Rosario, por la exposición al PM2.5 de los incendios en el Delta del río Paraná}. Ana Laura Fisanotti, Gamaliel López-Padilla, Jorge Zavatti y Adriana Ipiña. 13-14 de octubre de 2022, Rosario, Argentina. Premiado
\end{itemize}

LXV Congreso Nacional de Física
\begin{itemize}[label={}]
    \item \textit{Incendios en la Sierra Madre Oriental en la última década: récord en el año 2021 y su efecto en la calidad de aire en Monterrey}. Gamaliel López-Padilla, Martín Freire, Karen López-Cárdenas, Rubén D. Piacentini, Adriana Ipiña 2 al 7 de octubre 2022, Zacatecas, México.
\end{itemize}

XXXII Congreso Científico Argentino de Estudiantes de Medicina (CoCAEM) -XIX Jornadas AECUBA
\begin{itemize}[label={}]
    \item \textit{Influencia de la contaminación producida por los incendios del delta del río paraná sobre los tiempos de exposición solar para la síntesis de vitamina D3}. Ana Laura Fisanotti, Gamaliel López-Padilla, Silvana Spinelli y Adriana Ipiña Hernández. 24-27 noviembre 2021, Universidad de Buenos Aires, Argentina. Premiado
\end{itemize}

XV Jornadas de Ciencias, Tecnologías e Innovación de la UNR
\begin{itemize}[label={}]
    \item \textit{Condiciones meteorológicas invernales en el periodo 2010-2020 y su influencia en el incremento de incendios en el Delta del río Paraná}. Florencia Gómez-Fava, Adelina Navarro, Gamaliel López Padilla y Adriana Ipiña. 17-19 noviembre 2021, Rosario, Argentina. Webinar
\end{itemize}

XXXI Jornadas Científicas Anuales de la Asociación Científica Rosarina de Estudiantes de la Salud (ACRES)
\begin{itemize}[label={}]
    \item \textit{Influencia de la contaminación producida por los incendios del delta del río paraná sobre los tiempos de exposición solar para la síntesis de vitamina D3}. Ana Fisanotti, Gamaliel López-Padilla y Adriana Ipiña 13-14 de octubre de 2021, Rosario, Argentina.
\end{itemize}

LXIV Congreso Nacional de Física
\begin{itemize}[label={}]
    \item \textit{Herramienta computacional para el conteo de focos de incendios en dos regiones: la Sierra  Madre Oriental (México) y el Delta del río Paraná (Argentina)}. Gamaliel López-Padilla, Adriana Ipiña. LXIV-003944
    \item \textit{Anomalías meteorológicas y de la concentración de PM2.5 durante la temporada de incendios en el Delta del río Paraná, Argentina}. Gamaliel López-Padilla, Florencia Gómez-Fava, Adelina Navarro, Adriana Ipiña, Rubén Piacentini. LXIV-003956
    \item \textit{Influencia de las condiciones meteorológicas en la propagación de los incendios forestales ocurridos entre marzo y abril 2021 en las Sierras de Coahuila y Nuevo León, México}. Karen Giselle López-Cárdenas, Adriana Ipiña, Gamaliel López-Padilla, Rubén D. Piacentini. LXIV-4234
    \item \textit{Efecto de la suspensión de actividades por COVID-19 sobre la concentración de PM10 en el área metropolitana de Monterrey, México}. Constanza Zúñiga, Adriana Ipiña, Gamaliel López-Padilla, Jair Rafael Carrillo Avila, Rubén D Piacentini. LXIV-4301
\end{itemize}

106a Reunión de la Asociación Física Argentina
\begin{itemize}[label={}]
    \item \textit{Caracterización de imágenes satelitales de dos campos de exterminio en el noreste de México}. Gamaliel López-Padilla, María Luisa Castellanos-López y Adriana Ipiña.
    \item \textit{Años 2020 y 2021: récords de incendios forestales en regiones de Argentina y México}. Karen Giselle López-Cárdenas, Adriana Ipiña, Gamaliel López-Padilla, Rubén D. Piacentini.
    \item \textit{Estudio de la contribución de fuentes móviles en el PM 10 medido en Monterrey durante la suspensión de actividades por COVID-19}. Constanza Zúñiga, Adriana Ipiña, Gamaliel López-Padilla, Jair Rafael Carrillo Avila, Rubén D. Piacentini.
    \item \textit{Anomalías meteorológicas en el periodo 2020-2021: influencias en el incremento de incendios en el Delta del río Paraná}. Florencia Gómez-Fava, Adelina Navarro, Gamaliel López-Padilla, Adriana Ipiña.
\end{itemize}

LXIII Congreso Nacional de Física
\begin{itemize}[label={}]
    \item \textit{Desarrollo de la plataforma ‘TES para tu salud’  que determina los Tiempos de Exposición Solar adecuados para el tratamiento de Psoriasis en la Ciudad de México}. Gamaliel López-Padilla, Adriana Ipiña, Rubén D. Piacentini.
    \item \textit{Análisis del PM10 medido y el AOD550nm estimado a partir de las mediciones de irradiancia solar VIS-NIR en el Área Metropolitana de Monterrey}. Gamaliel López-Padilla, Adriana Ipiña, Rubén D. Piacentini.
\end{itemize}

105a Reunión de la Asociación Física Argentina
\begin{itemize}[label={}]
    \item \textit{Análisis del AOD550nm medido por el instrumento satelital MODIS/NASA y estimado a partir de las mediciones de irradiancia solar UV+VIS+NIR en el Área Metropolitana de Monterrey, México}. Gamaliel López-Padilla, Adriana Ipiña, Constanza Zúñiga, Rubén D. Piacentini.
    \item \textit{Análisis de los incendios detectados satelitalmente en la región del Delta del río Paraná frente a Rosario, en el periodo Junio-Agosto 2020}. Adriana Ipiña, Gamaliel López, Rubén D. Piacentini.
    \item \textit{Comparación de tres métodos de derivación de la irradiancia solar efectiva para la producción de pre-vitamina D en la piel, en la ciudad de Rosario, Argentina}. Montserrat Dávalos, Adriana Ipiña, Gamaliel López-Padilla, Rubén D. Piacentini.
\end{itemize}

VI Jornadas de Física Aplicada a las Ciencias Biomédicas, Rodolfo J. Rasia
\begin{itemize}[label={}]
    \item \textit{Determinación de los tiempos de exposición solar para el tratamiento de Psoriasis}. Adriana Ipiña, Giovanni Gamaliel López, Rubén D. Piacentini. Disertación. CCT-Rosario, Rosario, Argentina, 28-29 de Noviembre 2019.
\end{itemize}

LXII Congreso Nacional de Física
\begin{itemize}[label={}]
    \item \textit{Estudio de la irradiancia solar con dos modelos de transferencia radiativa y su comparación con mediciones en la ciudad de Monterrey}. Gamaliel López, Adriana Ipiña, Martin Freire, Benedetto Schiavo, Rubén Piacentini.
    \item \textit{Determinación de los tiempos de exposición solar para el tratamiento de Psoriasis en la ciudad de Monterrey}. Gamaliel López-Padilla, Adriana Ipiña, Rubén D Piacentini.
\end{itemize}

104a Reunión de la Asociación Física Argentina
\begin{itemize}[label={}]
    \item \textit{Análisis de las irradiancias UVA y Eritémica medidas por el Sistema Monitoreo Atmosférico de la Ciudad de México}. Adriana Ipiña, Gamaliel López, Rubén Piacentini.
    \item \textit{Estimación del Índice UV a partir de mediciones de intensidad solar espectral e irradiancia UVA+UVB en Ciudad de México}. Adriana Ipiña, Benedetto Schiavo, Gamaliel López, Martin Freire, Rubén Piacentini.
\end{itemize}

2da Reunión de la Red Temática de Contaminación Atmosférica y Mitigación del Cambio Climático y 8vo Congreso Nacional de Investigación en Cambio Climático.
\begin{itemize}[label={}]
    \item \textit{Modelización de la irradiancia solar UV-VIS-NIR* en el área metropolitana de la ciudad de Monterrey, México}. Giovanni Gamaliel López Padilla, Adriana Ipiña. 8-12 octubre 2018, Ciudad de México, México.
\end{itemize}

\textbf{Reportes para la Plataforma de Estudios Ambientales y Sostenibilidad  de la UNR (2)}
\begin{itemize}[label={}]
    \item \textit{Impacto en la calidad del aire en la ciudad de Rosario por la quema de pastizales en el Delta del río Paraná, agosto 2022}. Ipiña A, Piacentini R.D., Bolmaro R. López-Padilla G. \url{http://hdl.handle.net/2133/24201}
    \item \textit{Impacto de las emisiones derivadas de la quema de pastizales en el Delta del río Paraná Jun-Ago2022}. Ipiña A, López-Padilla G. \url{http://hdl.handle.net/2133/24085}
\end{itemize}
\end{document}
